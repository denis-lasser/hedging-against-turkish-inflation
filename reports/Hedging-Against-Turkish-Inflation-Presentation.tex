\documentclass{beamer}

\usetheme{Madrid}
\setbeamertemplate{navigation symbols}{}
\setbeamertemplate{footline}[frame number]

\title{Hedging Against Turkish Inflation}
\author{Selin Acar, Necati Furkan Çolak, Denis Lasser}
\date{December 2024}

% Bibliography setup
\usepackage[round]{natbib}
\bibliographystyle{plainnat}
\setbeamertemplate{bibliography item}[text]

\begin{document}

% Title Page
\frame{\titlepage}

% Table of Contents
\begin{frame}{Outline}
\tableofcontents
\end{frame}

% Introduction I
\section{Introduction}
\begin{frame}{Introduction I}
\begin{itemize}
\tightlist
\item Ongoing economic crisis in Türkiye has been characterized by
\textbf{high inflation}, the \textbf{depreciation of the Turkish Lira}
(TRY), \textbf{rising borrowing costs}, and \textbf{increasing loan
defaults}.
\item The underlying causes of the crisis can be attributed to
\textbf{political instability}, \textbf{global economic pressures},
and the implementation of \textbf{unconventional economic policies}.
\item Since 2018, inflation has been a persistent issue, partly due to
fluctuations in global oil prices and exchange rates, with
\textbf{Turkey's inflation rate ranking among the highest in emerging
markets} \citep{yilmazkuday_2022}.
\end{itemize}
\end{frame}

% Introduction II
\begin{frame}{Introduction II}
\begin{itemize}
\tightlist
\item The Turkish government's policy of maintaining low interest rates to
stimulate growth, despite the conventional theory that raising
interest rates should reduce inflation, has contributed significantly
to the current economic challenges \citep{kantur_ozcan_2021}.
\item Interest rates have been raised significantly. The most recent rate
hike was in March 2024, when the central bank increased the
\textbf{interest rate} by 5 percentage points to \textbf{50\%} as the
year-on-year inflation reached 75\% \citep{bloomberg_2024}.
\item Since then, the rate has remained unchanged as the central bank has
been committed to fighting high inflation and stabilising the Turkish
lira. As of October 2024, \textbf{year-on-year inflation stands at
48\%} \citep{bloomberg_2024}.
\end{itemize}
\end{frame}

% Methodology I
\section{Methodology}
\subsection{Analyze Asset Performance}
\begin{frame}{Methodology I: Analyze Asset Performance}
\begin{itemize}
\tightlist
\item Analyze asset performance using monthly data (Jan 2018--Sep 2024).
\item Two steps:
\begin{enumerate}
\item Evaluate individual asset hedging effectiveness.
\item Construct an optimal hedging portfolio using Markowitz optimization.
\end{enumerate}
\end{itemize}
\end{frame}

% Methodology II
\subsection{Data Sources}
\begin{frame}{Methodology II: Data Sources}
\begin{itemize}
\tightlist
\item \textbf{TUCXEFYY Index}: Turkish core CPI (excluding food, energy,
alcohol, tobacco).
\item \textbf{XU100 Index}: Turkish stock index (BIST 100).
\item \textbf{GTTRY10YR}: 10-Year Turkish Government Bond.
\item \textbf{XAU (Gold in USD)}: Global inflation hedge.
\item \textbf{XBT BGN (Bitcoin in USD)}: Alternative hedge.
\item \textbf{TP KFE TR-1}: Turkish house price index (real estate sector).
\end{itemize}
\end{frame}

% Methodology III
\subsection{Excess Return}
\begin{frame}{Methodology III: Excess Return}
\[
\text{Excess Return} = \text{Nominal Return} - \text{Inflation Rate}
\]

\[
\text{Excess Return} = \frac{1 + \text{Nominal Return}}{1 + \text{Inflation Rate}} - 1
\]
\end{frame}

% Methodology IV
\subsection{Markowitz Optimization}
\begin{frame}{Methodology IV: Markowitz Optimization}
\begin{itemize}
\item Portfolio return: \[
R_p = \sum_{i=1}^{n} w_i R_i
\]
\item Portfolio variance: \[
\sigma_p^2 = \sum_{i=1}^{n} \sum_{j=1}^{n} w_i w_j \sigma_{ij}
\]
\item Minimize portfolio risk: \[
\min_{w} \, \sigma_p^2 \quad \text{subject to:} \quad \sum_{i=1}^{n} w_i R_i = R_{\text{target}}, \quad \sum_{i=1}^{n} w_i = 1, \quad w_i \geq 0
\]
\end{itemize}
\end{frame}

% Results: Individual Assets
\section{Results}
\begin{frame}{Results: Turkish Stocks}
\begin{itemize}
\tightlist
\item Discuss performance of Turkish stocks as an inflation hedge.
\end{itemize}
\end{frame}

\begin{frame}{Results: Turkish Government Bond}
\begin{itemize}
\tightlist
\item Discuss performance of Turkish government bonds as an inflation hedge.
\end{itemize}
\end{frame}

\begin{frame}{Results: Gold}
\begin{itemize}
\tightlist
\item Discuss performance of gold as an inflation hedge.
\end{itemize}
\end{frame}

\begin{frame}{Results: Bitcoin}
\begin{itemize}
\tightlist
\item Discuss performance of Bitcoin as an inflation hedge.
\end{itemize}
\end{frame}

\begin{frame}{Results: Turkish Real Estate}
\begin{itemize}
\tightlist
\item Discuss performance of Turkish real estate as an inflation hedge.
\end{itemize}
\end{frame}

% Results: Markowitz Optimization
\begin{frame}{Results: Markowitz Optimization}
\begin{itemize}
\tightlist
\item Discuss portfolio composition and its performance.
\end{itemize}
\end{frame}

% Conclusion
\section{Conclusion}
\begin{frame}{Conclusion}
\begin{itemize}
\tightlist
\item Summary of findings.
\item Best hedging assets and portfolio configuration.
\item Implications for policy and future research.
\end{itemize}
\end{frame}

% Resources
\section{References}
\begin{frame}[allowframebreaks]{References}
\bibliography{references} % Ensure references.bib is in the same directory or update the path.
\end{frame}

\end{document}


