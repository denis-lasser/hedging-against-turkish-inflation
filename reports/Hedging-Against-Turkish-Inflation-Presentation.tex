% Options for packages loaded elsewhere
\PassOptionsToPackage{unicode}{hyperref}
\PassOptionsToPackage{hyphens}{url}
%
\documentclass[
  11pt,
]{article}
\usepackage{amsmath,amssymb}
\usepackage{iftex}
\ifPDFTeX
  \usepackage[T1]{fontenc}
  \usepackage[utf8]{inputenc}
  \usepackage{textcomp} % provide euro and other symbols
\else % if luatex or xetex
  \usepackage{unicode-math} % this also loads fontspec
  \defaultfontfeatures{Scale=MatchLowercase}
  \defaultfontfeatures[\rmfamily]{Ligatures=TeX,Scale=1}
\fi
\usepackage{lmodern}
\ifPDFTeX\else
  % xetex/luatex font selection
\fi
% Use upquote if available, for straight quotes in verbatim environments
\IfFileExists{upquote.sty}{\usepackage{upquote}}{}
\IfFileExists{microtype.sty}{% use microtype if available
  \usepackage[]{microtype}
  \UseMicrotypeSet[protrusion]{basicmath} % disable protrusion for tt fonts
}{}
\makeatletter
\@ifundefined{KOMAClassName}{% if non-KOMA class
  \IfFileExists{parskip.sty}{%
    \usepackage{parskip}
  }{% else
    \setlength{\parindent}{0pt}
    \setlength{\parskip}{6pt plus 2pt minus 1pt}}
}{% if KOMA class
  \KOMAoptions{parskip=half}}
\makeatother
\usepackage{xcolor}
\usepackage[margin=1in]{geometry}
\usepackage{graphicx}
\makeatletter
\def\maxwidth{\ifdim\Gin@nat@width>\linewidth\linewidth\else\Gin@nat@width\fi}
\def\maxheight{\ifdim\Gin@nat@height>\textheight\textheight\else\Gin@nat@height\fi}
\makeatother
% Scale images if necessary, so that they will not overflow the page
% margins by default, and it is still possible to overwrite the defaults
% using explicit options in \includegraphics[width, height, ...]{}
\setkeys{Gin}{width=\maxwidth,height=\maxheight,keepaspectratio}
% Set default figure placement to htbp
\makeatletter
\def\fps@figure{htbp}
\makeatother
\setlength{\emergencystretch}{3em} % prevent overfull lines
\providecommand{\tightlist}{%
  \setlength{\itemsep}{0pt}\setlength{\parskip}{0pt}}
\setcounter{secnumdepth}{-\maxdimen} % remove section numbering
\ifLuaTeX
  \usepackage{selnolig}  % disable illegal ligatures
\fi
\usepackage{bookmark}
\IfFileExists{xurl.sty}{\usepackage{xurl}}{} % add URL line breaks if available
\urlstyle{same}
\hypersetup{
  pdftitle={Hedging Against Turkish Inflation},
  pdfauthor={Selin Acar, Necati Furkan Çolak, Denis Lasser},
  hidelinks,
  pdfcreator={LaTeX via pandoc}}

\title{Hedging Against Turkish Inflation}
\author{Selin Acar, Necati Furkan Çolak, Denis Lasser}
\date{December 2024}

\begin{document}
\maketitle

\section{Introduction I}\label{introduction-i}

\begin{itemize}
\tightlist
\item
  Ongoing economic crisis in Türkiye has been characterized by
  \textbf{high inflation}, the \textbf{depreciation of the Turkish Lira}
  (TRY), \textbf{rising borrowing costs}, and \textbf{increasing loan
  defaults}.
\item
  The underlying causes of the crisis can be attributed to
  \textbf{political instability}, \textbf{global economic pressures},
  and the implementation of \textbf{unconventional economic policies}.
\item
  Since 2018, inflation has been a persistent issue, partly due to
  fluctuations in global oil prices and exchange rates, with
  \textbf{Turkey's inflation rate ranking among the highest in emerging
  markets} (Yilmazkuday, 2022).
\end{itemize}

\section{Introduction II}\label{introduction-ii}

\begin{itemize}
\tightlist
\item
  The Turkish government's policy of maintaining low interest rates to
  stimulate growth, despite the conventional theory that raising
  interest rates should reduce inflation, has contributed significantly
  to the current economic challenges (Kantur \& Özcan, 2022).
\item
  Interest rates have been raised significantly. The most recent rate
  hike was in March 2024, when the central bank increased the
  \textbf{interest rate} by 5 percentage points to \textbf{50\%} as the
  year-on-year inflation reached 75\% (Bloomberg, 2024).
\item
  Since then, the rate has remained unchanged as the central bank has
  been committed to fighting high inflation and stabilising the Turkish
  lira. As of October 2024, \textbf{year-on-year inflation stands at
  48\%} (Bloomberg, 2024).
\end{itemize}

\section{Methodology}\label{methodology}

\begin{itemize}
\tightlist
\item
  Analyze asset performance using monthly data (Jan 2018--Sep 2024).
\item
  Two steps:

  \begin{enumerate}
  \def\labelenumi{\arabic{enumi}.}
  \tightlist
  \item
    Evaluate individual asset hedging effectiveness.
  \item
    Construct an optimal hedging portfolio using Markowitz optimization.
  \end{enumerate}
\end{itemize}

\section{Methodology}\label{methodology-1}

\subsection{Data Sources}\label{data-sources}

\begin{itemize}
\tightlist
\item
  \textbf{TUCXEFYY Index}: Turkish core CPI (excluding food, energy,
  alcohol, tobacco).
\item
  \textbf{XU100 Index}: Turkish stock index (BIST 100).
\item
  \textbf{GTTRY10YR}: 10-Year Turkish Government Bond.
\item
  \textbf{XAU (Gold in USD)}: Global inflation hedge.
\item
  \textbf{XBT BGN (Bitcoin in USD)}: Alternative hedge.
\item
  \textbf{TP KFE TR-1}: Turkish house price index (real estate sector).
\end{itemize}

\section{Analysis Techniques}\label{analysis-techniques}

\subsection{Excess Return}\label{excess-return}

\begin{itemize}
\item
  TRY-denominated assets: \[
  \text{Excess Return} = \text{Nominal Return} - \text{Inflation Rate}
  \]
\item
  USD-denominated assets: \[
  \text{Excess Return} = \frac{1 + \text{Nominal Return}}{1 + \text{Inflation Rate}} - 1
  \]
\end{itemize}

\section{Analysis Techniques}\label{analysis-techniques-1}

\subsection{Markowitz Optimization}\label{markowitz-optimization}

\begin{itemize}
\item
  Portfolio return: \[
  R_p = \sum_{i=1}^{n} w_i R_i
  \]
\item
  Portfolio variance: \[
  \sigma_p^2 = \sum_{i=1}^{n} \sum_{j=1}^{n} w_i w_j \sigma_{ij}
  \]
\item
  Minimize portfolio risk: \[
  \min_{w} \, \sigma_p^2 \quad \text{subject to:} \quad \sum_{i=1}^{n} w_i R_i = R_{\text{target}}, \quad \sum_{i=1}^{n} w_i = 1, \quad w_i \geq 0
  \]
\end{itemize}

\section{Results}\label{results}

\begin{itemize}
\tightlist
\item
  Key findings on individual asset performance.
\item
  Optimized portfolio composition.
\end{itemize}

\section{Conclusion}\label{conclusion}

\begin{itemize}
\tightlist
\item
  Summary of findings.
\item
  Best hedging assets and portfolio configuration.
\item
  Implications for policy and future research.
\end{itemize}

\end{document}
